\chapter{Proceso}

\section{El Seguro}	

\subsection{Funciones del Seguro}
\setlength{\parskip}{5mm}

\setlength{\parskip}{0mm}
	

\subsection{Seguro de Vehiculos}
\setlength{\parskip}{5mm}

\setlength{\parskip}{0mm}

\subsection{El Perito}
\setlength{\parskip}{5mm}
	El Perito es esencial en el engranaje de la compañía de seguros, pero para conocer la verdadera dimensión del trabajo del perito, analizamos sus funciones, que se resumen en tres grandes apartados:


\subsubsection{Aspectos técnicos}
\begin{itemize}

\item Valoración económica de los daños, elaborando la peritación y realizando la propuesta de indemnización a la compañía de seguros. Determinación del valor del bien asegurado, como, por ejemplo, el
valor venal, el valor de mercado, el valor de los restos y la propuesta del importe líquido de la indemnización, cuando se ha producido un siniestro total o una pérdida total.

\item Verificación de siniestros, para la realización de informes de uso interno para la compañía de seguros con la justificación técnica de la ocurrencia del siniestro. Pueden ser informes de rehúses parciales o totales, que pueden aportarse como prueba en un juicio. Los informes de reconstrucción de accidentes de tráfico, a partir de huellas y vestigios, mediante cálculos físicos y matemáticos, pueden ser también un apoyo para la determinación de la culpabilidad en el juicio. 

\item Revisión de riesgos, para la contratación de nuevas pólizas de vehículos de segunda mano con coberturas de daños propios, lunas, etc.

\item Control de calidad de la reparación, mediante la comprobación, en primer lugar, de que la reparación se ha llevado conforme a la peritación en todas y cada una de las partidas asignadas por el perito; a continuación, que la reparación se ha realizado con las debidas garantías técnicas, de calidad y seguridad para los ocupantes del vehículo. Por último, se analizarán los defectos en la reparación, para que sean subsanados por el taller.

\item Averías mecánicas: valoración y peritación de los daños mecánicos bajo la cobertura de pólizas de vehículos de renting y de pólizas de garantía de venta de vehículos usados.

\end{itemize}


\subsubsection{Aspectos administrativo-legales}
\begin{itemize}

\item Implicación en la tramitación del siniestro. El perito, en contacto con el tramitador y a través del sistema de gestión de la compañía de seguros, está al día de la tramitación de los siniestros, del tipo de pólizas que comercializa la compañía de seguros, de sus coberturas y exclusiones, de los convenios entre compañías y del conocimiento de la legislación de seguros.

\end{itemize}

\subsubsection{Aspecto negociador}
\begin{itemize}

\item El perito es la imagen de la compañía de seguros, ya que está en contacto con los asegurados, perjudicados, talleres, otras compañías, con lo que su actuación está sujeta a examen continuo, y su comportamiento, a ojos del asegurado, es, por extensión, el de la compañía de seguros. 

\item El perito debe aportar, en todo momento, argumentos y criterios técnicos en la negociación con el taller.

\item Ha de consensuar la peritación: debe llegar a acuerdos con el taller sobre todas y cada una de las partidas que componen una peritación.

\item Realiza asesoría legal: al estar en contacto con los asegurados y el taller, en muchas ocasiones, el perito se convierte en el asesor sobre los aspectos legales de los siniestros

\end{itemize}


(\citet{peritobib}, 2012)
\setlength{\parskip}{0mm}