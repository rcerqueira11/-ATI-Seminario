\chapter{Propuesta de trabajo de grado}

\section{Planteamiento del problema}
\setlength{\parskip}{5mm}
% Como se vio en la Sección anterior, actualmente existen distintas implementaciones del juego Rithmomachia para diversos dispositivos, pero ninguno provee la posibilidad de poder hacer uso de las reglas del Club Venezolano de Rithmomachia, que contenga la capacidad de poder gestionar un torneo local.

Actualmente al asistir a la cita para la inspección del carro, se anotan las especificaciones del mismo en una hoja para luego ser pasada a la computadora, este proceso ocasiona gastos de recursos tanto de personal como "de materiales" como hojas entre otras  

Recibirá mediante un web service la información de la planilla y que esta pueda ser editada una vez antes de aceptar y guardarla en el sistema
\setlength{\parskip}{0mm}


\section{Justificación}

% Este trabajo se justifica en la posibilidad de apoyar al Club Venezolano de Rithmomachia permitiendo que se puedan realizar actividades interactivas por medio de la computación, además de llevar a cabo torneos autogestionados, permitiendo de esta manera la posibilidad de almacenar de forma automática un registro de las partidas jugadas y así poder usar esta información para futuros trabajos e investigaciones sobre el juego. 


\section{Objetivo general}

Desarrollar una aplicación con tecnología móvil multiplataforma y una aplicación con tecnología web, que permita sistematizar la planilla de inscripción para ser utilizada por el personal de una empresa de Seguros. 

\section{Objetivos específicos}

\begin{itemize}

	\item Diseñar e implementar un sistema de registro.
	
	\item Diseñar e implementar un sistema de inscripción mediante dispositivos móviles.
	
	\item Diseñar e implementar las base de datos centrar y la base de datos móvil.
	
	\item Permitir la conexión y sincronización de la aplicación móvil con la aplicación web.
	
	\item Permitir la edición de la solicitudes de inscripción mediante la aplicación web.
	
	

\end{itemize}



\subsection{Solución Propuesta}
\setlength{\parskip}{5mm}
Se propone el uso de un sistema web con la finalidad de mantener la persistencia de las planilla de registro del seguro digitalmente, y facilitar su acceso y edición de las mismas. Este sistema web estará compuesto de varios módulos, un modulo de registro donde el personal de la compañía encargado de la gestión de estas solicitudes podrá registrarse para tener acceso al segundo modulo que es el de edición y aprobación de las solicitudes de inscripción de la compañía de seguros.



Para la base de datos utilizaremos PostgreSQL, ya que es una de las bases de datos mas potentes de software libre, con un alto rendimiento, seguridad y a su vez estar disponible prácticamente para todas las versiones de los sistemas operativos Unix y también Windows. Su versatilidad y robustez la hace el candidato perfecto para el sistema que se busca implementar.

Se propone utilizar el framework de Django para el desarrollo del sistema. Ya que siendo uno de los muchos frameworks que están establecidos sobre la base del patrón MVC en sus beneficios se encuentra la separación de responsabilidad y organización del codigo. Este framework incorpora un patrón llamado MTV (Modelo-Template-Vista), los templates son donde se implementan todas las interfaces de los usuarios que serán desplegadas por las vistas en el lado del servidor y los models corresponde a la base da datos donde persistirá la información. Además posee un sistema jerárquico de plantilla que proporciona la reutilización de código y la extensibilidad de las aplicaciones. Posee un buen soporte para PostgresSQL, la base de datos que se piensa utilizar para el sistema. 
\setlength{\parskip}{0mm}



\subsection{Metodología de desarrollo a utilizar}


\setlength{\parskip}{5mm}
Para el desarrollo de este proyecto, se quiere utilizar la metodología Scrum, ya que es una de las metodologías mas usadas actualmente en el mercado para la realización de proyectos. Esta metodología aplica de manera regular un conjunto de mejores prácticas para trabajar en equipo y obtener el mejor resultado posible de un proyecto.

Es esta metodología se realizan entregas parciales del resultado final del proyecto, priorizadas por el beneficio que aportan al receptor. Cada interacción permite evaluar el desempeño de las funcionalidades asociadas al sistema. Posee la flexibilidad y adaptación de trabajar respecto a las necesidades del cliente. 

\setlength{\parskip}{0mm}

\section{Descripción del flujo asociado a la solución}


\section{Alcance}

% Como alcance del trabajo especial de grado se tiene previsto lograr que la interacción se logre únicamente con jugadores humanos, dejando de esta manera la implementación inteligencia artificial para futuros trabajos. En referencia a la gestión de torneos, te tomará en consideración únicamente la implementación de tipo llave (o bracket) permitiendo que la cantidad mínima de jugadores para poder hacer uso de esta funcionalidad, tiene que ser un numero regido por una sucesión geométrica, con factor = 2, y que empiece por el numero 4.  ------------ ojo con esta formula