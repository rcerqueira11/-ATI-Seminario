\chapter{Propuesta de trabajo de grado}

\section{Planteamiento del problema}

% Como se vio en la Sección anterior, actualmente existen distintas implementaciones del juego Rithmomachia para diversos dispositivos, pero ninguno provee la posibilidad de poder hacer uso de las reglas del Club Venezolano de Rithmomachia, que contenga la capacidad de poder gestionar un torneo local.

Actualmente al asistir a la cita para la inspeccion del carro, se anotan las especificaciones del mismo en una hoja para luego ser pasada a la computadora, este proceso ocasiona gastos de recursos tanto de personal como "de materiales" como hojas entre otras  

Recibira mediante un web service la informacion de la planilla y que esta pueda ser editada una vez antes de aceptar y guardarla en el sistema

\section{Objetivo general}

Desarrollar una aplicación con tecnología móvil multiplataforma y una aplicación con tecnología web, que permita sistematizar la planilla de inscripción para ser utilizada por el personal de una empresa de Seguros. 

\section{Objetivos específicos}

\begin{itemize}

	\item Diseñar e implementar un sistema de registro.
	
	\item Diseñar e implementar un sistema de inscripción mediante dispositivos móviles.
	
	\item Diseñar e implementar las base de datos centrar y la base de datos móvil.
	
	\item Permitir la conexión y sincronización de la aplicación móvil con la aplicación web.
	
	\item Permitir la edición de la solicitudes de inscripción mediante la aplicación web.
	
	

\end{itemize}

\section{Justificación}

% Este trabajo se justifica en la posibilidad de apoyar al Club Venezolano de Rithmomachia permitiendo que se puedan realizar actividades interactivas por medio de la computación, además de llevar a cabo torneos autogestionados, permitiendo de esta manera la posibilidad de almacenar de forma automática un registro de las partidas jugadas y así poder usar esta información para futuros trabajos e investigaciones sobre el juego. 

\section{Consideraciones}

% Como alcance del trabajo especial de grado se tiene previsto lograr que la interacción se logre únicamente con jugadores humanos, dejando de esta manera la implementación inteligencia artificial para futuros trabajos. En referencia a la gestión de torneos, te tomará en consideración únicamente la implementación de tipo llave (o bracket) permitiendo que la cantidad mínima de jugadores para poder hacer uso de esta funcionalidad, tiene que ser un numero regido por una sucesión geométrica, con factor = 2, y que empiece por el numero 4.  ------------ ojo con esta formula

\subsection{Soluci&oacute;n Propuesta}

se utilizara django
utilizara postgres

\subsection{Metodología de desarrollo a utilizar}

Utilizare scrum blablabla

Para el desarrollo de este proyecto, se quiere utilizar la metodologia Scrum, ya que es una de las metodologias mas usadas actualmente en el mercado para la realizacion de proyectos. Esta metodologia aplica de manera regular un conjunto de mejores prácticas para trabajar en equipo y obtener el mejor resultado posible de un proyecto.

Es esta metodologia se realizan entregas parciales del resultado final del proyecto, priorizadas por el beneficio que aportan al receptor. Cada interaccion permite evaluar el desempe;o de las funcionalidades asociadas al sistema y posee la flexibilidad de trabajar con requisitos inestables.









\section{Plan de trabajo}
