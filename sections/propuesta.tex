\chapter{Propuesta de trabajo de grado}

\section{Planteamiento del problema}

% Como se vio en la Sección anterior, actualmente existen distintas implementaciones del juego Rithmomachia para diversos dispositivos, pero ninguno provee la posibilidad de poder hacer uso de las reglas del Club Venezolano de Rithmomachia, que contenga la capacidad de poder gestionar un torneo local.

\section{Objetivo general}

Desarrollar una aplicación con tecnología móvil multiplataforma y una aplicación con tecnología web, que permita sistematizar la planilla de inscripción para ser utilizada por el personal de una empresa de Seguros. 

\section{Objetivos específicos}

\begin{itemize}

	\item Diseñar e implementar un sistema de registro.
	
	\item Diseñar e implementar un sistema de inscripción mediante dispositivos móviles.
	
	\item Diseñar e implementar las base de datos centrar y la base de datos móvil.
	
	\item Permitir la conexión y sincronización de la aplicación móvil con la aplicación web.
	
	\item Permitir la edición de la solicitudes de inscripción mediante la aplicación web.
	
	

\end{itemize}

\section{Justificación}

% Este trabajo se justifica en la posibilidad de apoyar al Club Venezolano de Rithmomachia permitiendo que se puedan realizar actividades interactivas por medio de la computación, además de llevar a cabo torneos autogestionados, permitiendo de esta manera la posibilidad de almacenar de forma automática un registro de las partidas jugadas y así poder usar esta información para futuros trabajos e investigaciones sobre el juego. 

\section{Consideraciones}

% Como alcance del trabajo especial de grado se tiene previsto lograr que la interacción se logre únicamente con jugadores humanos, dejando de esta manera la implementación inteligencia artificial para futuros trabajos. En referencia a la gestión de torneos, te tomará en consideración únicamente la implementación de tipo llave (o bracket) permitiendo que la cantidad mínima de jugadores para poder hacer uso de esta funcionalidad, tiene que ser un numero regido por una sucesión geométrica, con factor = 2, y que empiece por el numero 4.  ------------ ojo con esta formula

\subsection{Metodología de desarrollo y patrones de diseño}

\subsection{Tecnologías a utilizar}

\section{Plan de trabajo}
